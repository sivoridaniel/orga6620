\documentclass[a4paper,10pt]{article}
% Paquete para inclusión de gráficos.
\usepackage{graphicx}
% Paquete para definir el idioma usado.
\usepackage[spanish]{babel}
% Paquete para definir la codificación del conjunto de caracteres usado
% (latin1 es ISO 8859-1).
\usepackage[latin1]{inputenc}
\usepackage{hyperref}
% Include the listings-package
\usepackage{listings}  
\usepackage{pdfpages}


% Título principal del documento.
\title{	\ Trabajo Práctico Nº 0: Infraestructura Básica}
% Información sobre los autores.
\author{	Sebastian Ripari, \textit{Padrón Nro. 96453}\\
            \texttt{sebastiandanielripari@hotmail.com }\\\\
            Cesar Emanuel Lencina, \textit{Padrón Nro. 96078}\\
            \texttt{cel_1990@live.com}\\\\
			Pablo Sivori, \textit{Padrón Nro. 84026}\\
            \texttt{sivori.daniel@gmail.com}\\\\               
            \texttt{\footnotesize 1º Entrega: 07/09/2017}\\
            \\\\\\\\\\\\\\\\\\
            \normalsize{2do. Cuatrimestre de 2017}\\ 
            \normalsize{66.20 Organización de Computadoras} \\
            \normalsize{Facultad de Ingeniería, Universidad de Buenos Aires} \\}
       
\date{}

\begin{document}
% Inserta el título.
\maketitle
% quita el número en la primer página
\thispagestyle{empty}
% Resumen
\begin{abstract}
En el presente trabajo práctico se describirán todos los pasos y 
conclusiones relacionadas al desarrollo e implementación de una versión en lenguaje C,
de la Criba de Eratóstenes. Y su posterior compilacion y ejecucion bajo el procesador MIPS.
\end{abstract}
\newpage{}
\tableofcontents
\newpage{}

\begin{flushleft}

\par\end{flushleft}
\section{{\normalsize Introducción}}

El objetivo del presente trabajo práctico es familiarizarse con el emulador gxemul mediante la implementacion de un programa que nos devuelva por stdout o en un archivo de salida, los números primos menores a un número natural N el cuál es ingresado por parámetro.

\section{{\normalsize Entorno de trabajo}}
Mediante el emulador Gxemule pudimos simular una DEC Station 5000/200 con un microprocesador MIPS de 32 Bits, corriendo un sistema 
operativo NetBSD/pmax.

\subsection{{\normalsize Lenguaje}}

Como lenguaje de implementación se eligio C ya que el mismo permite una alta portabilidad entre 
diferentes plataformas. El desarrollo del programa se realizó usando editores de texto 
(gedit,vim, kwrite y sublime) y compilando los archivos fuente con 
\htmladdnormallink{GCC}{http://gcc.gnu.org/} que viene en linux. Ya que esta compilador es compatible
con el sistema operativo NetBSD y con la arquitecura MIPS.
Para compilar, ejecutar el siguiente comando:

\begin{tabbing}
------- \= ----- \= \kill
\> \textbf{\emph{\$ make}}\\ 
\end{tabbing}

\subsection {{\normalsize Descripción del programa}}

Por empezar tenemos un archivo llamado \texttt{erat.c} que contiene la funcion
\texttt{main}, aqui lo primero que hacemos es la validacion de los argumentos, mediante 
una funcion llamada \texttt{validarArgumentos}. Caso de ingresarse algo invalido 
el programa no continuara, y retornara un valor distinto de 0, osea un codigo de error. Caso contrario, 
se llama a la funcion \texttt{realizarAccion} y aqui comienza el procesamiento de calcular 
la cantidad de numeros primos, entre 2 y el numero \texttt{tope} ingresado. Basicamente lo que hacemos es 
crear un array que contiene todos los numeros entre 2 y el numero ingresado, el \texttt{tope}. 
y luego le aplicamos a este array el algoritmo de la Criba de Erastostenes, mediante la funcion \texttt{encontrarNumerosPrimos},
que en las posiciones del array donde no hay un numero primo setea un cero.
Para luego tomar este array desde una funcion que se llama \texttt{imprimirPorPantalla}, y aqui imprimir todos los numeros distintos
de cero, que son los primos.

\subsubsection {{\normalsize Errores posibles}}

\begin{enumerate}
\item El procesamiento de la entrada estándar causó el agotamiento del heap.
\item La invocación del programa es incorrecta.
\item El archivo de salida no se escribió correctamente.
\end{enumerate}

Se contemplan otros errores gracias al uso de la variable externa errno. 
Cuando ocurre un error inesperado, el mismo es informado por stderr y finaliza el programa
liberando la memoria que se habia solicitado hasta el momento.
(con la funión perror()).

\subsection{{\normalsize Desarrollo de actividades}}

\begin{enumerate}
\item Se instaló en un linux un repositorio de fuentes 
para que al dividir las tareas del TP se pudiese hacer una unión de
los cambios ingresados por cada uno de los integrantes más fácilmente. 
\item Cada persona del grupo se comprometió a que sus cambios en el código
fuente y los cambios obtenidos del repositorio que pudiesen haber subido los
otros integrantes del grupo, sean compilados los sistemas operativos Linux y el NetBSD, asegurando así portabilidad entre plataformas planteada en el enunciado. 
\item Se estableció que todos los integrantes en mayor o menor medida, 
contribuyan en el desarrollo de todas las partes del código para que 
nadie quede en desconocimiento de lo que se hizo en cada sección. 
Si bien cada parte del código fue realizada por diferentes integrantes (parseo
de los argumentos, lectura de los ficheros, etc), todos nos 
familiarizamos con cada una de estas partes y cumplimos la función 
de testers de lo hecho por otros integrantes. 
\item Para poder generar el código assembler a partir del código fuente, 
dentro de NETBSD se utilizó gcc con la siguiente opción: 
	\begin{tabbing}
	------- \= ----- \= \kill
	\> \textbf{\emph{gcc -S main.c}}\\ 
	\end{tabbing}
\item Para crear el presente informe se debe utilizar el comando make en el directorio informe.
\end{enumerate}

\newpage{}
\subsection{{\normalsize Corridas de pruebas}}
	Para correr las pruebas se debe ejecutar el comando make del directorio pruebas y se veran resultados como los de 
	a continuacion:
	\newline
	\begin{center}
		\includegraphics[width=110mm,height=80mm]{PONER GRAFICO}
	\end{center}	
	
\newpage
\section{{\normalsize El código fuente, en lenguaje C}}

	\lstinputlisting[language=C, basicstyle=\tiny]{../main.c}	

\newpage
\newpage
\newpage

\section{{\normalsize El código MIPS32 generado por el compilado}}

	\lstinputlisting[language={[x86masm]Assembler}, firstline=1, lastline=45, basicstyle=\small]{../main.s}
 
 
\newpage{}
\section{{\normalsize Conclusiones}}

\begin{enumerate}
\item Si bien lo solicitado por el programa no era excesivamente difícil,
la realización completa del TP llevó cierta dificultad al tener que
realizarlo en el contexto solicitado: alta portabilidad, desarrollo
en C, e informe hecho en LaTeX. 
\item En el primer caso la dificultad radicaba en tener configurado 
y funcionando el GXEmul dentro de un Linux, y lograr que en ambos casos 
el programa compile y corra sin problemas. 
\item Debido a nuestro desconocimiento con LaTeX, tuvimos que 
invertir tiempo en encontrar forma de realizar el presente documento 
de la manera más correcta posible 
\item En cuanto al trabajo grupal en si mismo, no hubo inconvenientes de
ningún tipo ya que al ser el grupo relativamente chico y tener conocimiento
del manejo del versionado de un proyecto ante cambios ingresado por
los integrantes (por medio del GIT), la introducción de modificaciones
y correcciones fué fluida. 
\end{enumerate}

\newpage
\section{{\normalsize Enunciado del trabajo practico}}

	%\includegraphics[width=0.8\textwidth]{enunciado.pdf}
	\includepdf[pages={1-},scale=0.75]{enunciado.pdf}


\bibliographystyle{plain}
\nocite{*}
\end{document}
